\documentclass[a4paper,10pt]{article}
\usepackage{cv,booktabs,fontawesome5}
\usepackage[utf8]{inputenc}
\usepackage[T1]{fontenc}
\usepackage{xcolor}
\usepackage[hidelinks]{hyperref}
\usepackage{filecontents}
\usepackage{longtable}

\LTpre=\smallskipamount
\LTpost=\smallskipamount
\LTleft=0pt % irrelevant, except it makes it easier to see the different spacing

% Change color to blue
\def\headcolor{\color[rgb]{0,0.3,0.5}}

% Space before section headings
\titlespacing{\section}{0pt}{2ex}{1ex}

\name{Rudolf Schlechter}

\info{\faIcon{map-marker-alt} & 4/589 Barbadoes street, \\
                          &  Christchurch 8013, New Zealand\\
      \faIcon{phone}      & +64 27 525 3637\\
      \faIcon{envelope}   & roschlec@gmail.com\\
      \faIcon{twitter}    & @rschlec\\
      \faIcon{github}     & \href{http://www.github.com/roschlec}{github.com/roschlec}\\
      \faIcon{orcid}      & \href{https://orcid.org/0000-0002-5717-2917}{0000-0002-5717-2917}\\
      \faIcon{graduation-cap} & \href{https://scholar.google.com/citations{?}user=D3zhLzEAAAAJ{&}hl=en}{Google Scholar (\textit{h}-index = 9.0)}\\
      \faIcon{globe-americas} & Chilean and German}

\DeclareSourcemap{
  \maps[datatype=bibtex, overwrite]{
    \map{
      \step[fieldset=month, null]
    }
  }
}

% Use useprefix because of Ashton de Silva
\ExecuteBibliographyOptions{useprefix=true}

\usepackage[sfdefault,lf,t]{carlito}
\renewcommand{\bibfont}{\normalfont\fontsize{10}{12.4}\sffamily}
\usepackage{inconsolata}


\addbibresource{rspub.bib}


\DeclareBibliographyCategory{Papers}
\DeclareBibliographyCategory{Preprints}
\DeclareBibliographyCategory{Books}

\DeclareIndexFieldFormat{Books}{%
  \ifstrequal{Books}{#1}
  {\addtocategory{Books}{\thefield{entrykey}}}
  {
    \ifstrequal{Papers}{#1}
    {\addtocategory{Papers}{\thefield{entrykey}}}
    {
      \ifstrequal{Preprints}{#1}
      {\addtocategory{Preprints}{\thefield{entrykey}}}
      {}
    }
  }
}
\AtDataInput{\indexfield[Books]{annotation}}

\DeclareNameAlias{sortname}{last-first}
\renewcommand*{\finalnamedelim}{\multinamedelim}


\begin{document}
\maketitle

\section{Research Interest}
\begin{itemize}\parskip=0cm
\item  My current research interests involve mechanisms that drive microbial community assembly in the phyllosphere. Combining culture-dependent techniques, molecular biology and genetics, fluorescence microscopy, spatial statistics, and macro- and microecological frameworks, I am interested in understanding bacterial interactions and their impact on microbial community structure and ecosystem functioning on plants.
\item Since 2013, I co-authored 12 peer-reviewed publications and 2 book chapters on topics related to plant biology
and microbiology. My research has been cited 602 times and both my h-index and i10-index are 9.0.
\end{itemize}

\section{Education}
~\begin{tabular}{p{3cm} p{14cm}} %

2017--Present & \textbf{University of Canterbury, Ph.D.} \textit{Microbiology}\\
 & {\small\textcolor{black!80}{Thesis: ``Driving factors of bacterial interactions and spatial patterns in the phyllosphere''}}\\
 & {\small\textcolor{black!80}{Senior Supervisor: Prof. M. Remus-Emsermann. Associate Supervisors: Prof. Emerita P. Jameson, Assoc. Prof. M. Stott}}\\

2013--2014 & \textbf{Pontificia Universidad de Chile, Licenciate} \textit{Biochemistry} (M.Sc. equivalent)\\
 & {\small\textcolor{black!80}{Thesis: ``Characterisation of the immune response conferred by the \textit{loci} \textit{RUN1} and \textit{REN1} in \textit{Vitis vinifera} against \textit{Erysiphe necator}''}}\\
 & {\small\textcolor{black!80}{Senior Supervisor: Prof. P. Arce-Johnson}}\\

2009--2013 & \textbf{Pontificia Universidad de Chile, B.Sc.} \textit{Biochemistry}
\end{tabular}

\section{Research Experience}
~\begin{longtable}{p{3cm} p{14cm}}
May 2017-Present & \textbf{University of Canterbury, Remus-Emsermann Lab}, Ph.D. student\\
& -- {\small\textcolor{black!80}{\textit{Bacterial genetic modification}: Development of a genetic toolbox for the stable labelling of \textit{Proteobacteria} with fluorescent proteins}}\\
& -- {\small\textcolor{black!80}{\textit{Genomics}: Reconstruction and analysis of genome-scale metabolic models to study metabolic relationships between phyllosphere bacteria}}\\
& -- {\small\textcolor{black!80}{\textit{Factors driving species interactions and spatial distribution patterns}: Combination of \textit{in vitro} and \textit{in planta} experiments with fluorescence microscopy and spatial statistics to determine the influence of resource overlap and phylogenetic relationships in interactions and spatial patterns between competing bacteria in the \textit{Arabidopsis thaliana} phyllosphere}}\\
& -- {\small\textcolor{black!80}{\textit{Single-cell bioreporters}: Use of CUSPER bioreporter in \textit{Pantoea eucalypti} 299R to estimate the effect of resource competition in single-cell bacterial fitness in the phyllosphere}}\\

Jan 2017--Apr 2017 & \textbf{Pontificia Universidad Católica de Chile, Arce-Johnson Lab}, Research Assistant\\
& -- {\small\textcolor{black!80}{\textit{Plant-pathogen interactions}: Gene expression analysis of grapevine response to the infection of powdery mildew}}\\
& -- {\small\textcolor{black!80}{\textit{Capillary electrophoresis}: Establishment and standardisation of DNA Genetic Analyser instrument for DNA fragment analysis through capillary electrophoresis}}\\

Oct 2015--Oct 2016 & \textbf{ETH, Plant Cell Biology Group}, Research Assistant\\
& -- {\small\textcolor{black!80}{\textit{Role of endocytosis in plant-pathogen interactions}: Live-cell imaging of \textit{Arabidopsis thaliana} root endocytic and vesicular trafficking response against \textit{Fusarium oxysporum}}}\\
& -- {\small\textcolor{black!80}{\textit{Calcium imaging}: Live-cell imaging of calcium dynamics in arabidopsis roots with the fluorescent biosensor R-GECO1 and the microfluidic platform RootChip. \textit{Visiting scholar at COS, University of Heidelberg. Schumacher and Grossmann Lab}}}\\

Apr 2014--Feb 2015 & \textbf{AgriJohnson Ltd.}, Research Assistant\\
& -- {\small\textcolor{black!80}{\textit{Grapevine virus detection platform}: Development of a simultaneous detection of grapevine viruses in \textit{Vitis vinifera} through multiplex PCR}}\\
& -- {\small\textcolor{black!80}{\textit{Tissue culture}: Establishment of virus-free commercially-relevant grapevine cultivars through plant tissue culture techniques}}\\

Jan 2013--May 2014 & \textbf{Pontificia Universidad Católica de Chile, Arce-Johnson Lab}, M.Sc. student\\
& -- {\small\textcolor{black!80}{\textit{Selection of resistant grapevine genotypes}: Use of molecular markers to select for grapevine individuals that carry the resistant \textit{loci} \textit{RUN1} and \textit{REN1}}}\\
Jan 2013--May 2014 & -- {\small\textcolor{black!80}{\textit{Cellular and molecular immune responses of resistant and susceptible grapevine varieties against powdery mildew}: The response of grapevine plants carrying the \textit{loci} \textit{RUN1} and/or \textit{REN1} were evaluated upon inoculation with the powdery mildew. Plants carrying both \textit{loci} were associated with the suppression of fungal spore germination and an increased expression of a gene related to stilbene biosynthesis, which is involved in plant biotic stress responses}}\\
\end{longtable}

\section{Academic Experience}

\subsection{Honours and Awards}

~\begin{tabular}{p{3cm} p{14cm}}
2019 & Travel grant - New Zealand Microbiological Society (NZMS) Conference\\
2018 & Third Place Student Poster Presentation Competition. NZMS-NZSBMB Joint Annual Conference. University of Otago, Dunedin, New Zealand\\
2017--2021 & New Zealand International Doctoral Research Scholarship. Education New Zealand (ENZ), New Zealand\\
2017 & UC College of Science PhD Scholarship. University of Canterbury, New Zealand\\
2011-2013 & Honour Scholarship for excellent academic performance. Pontificia Universidad Católica de Chile, Chile\\
\end{tabular}

\subsection{Teaching}
~\begin{tabular}{p{3cm} p{14cm}}
Semester 2 2021    & \textbf{University of Canterbury}, Guest Lecturer\\
& \small\textcolor{black!80}{BIOL313: Advanced Microbiology}\\

Semester 2 2020 & \textbf{University of Canterbury}, Demonstrator (Teaching assistant)\\
& \small\textcolor{black!80}{BIOL313: Advanced Microbiology}\\

Semester 2 2019 & \textbf{University of Canterbury}, Demonstrator (Teaching assistant)\\
& \small\textcolor{black!80}{BIOL313: Advanced Microbiology}\\
Semester 2 2018 & \textbf{ University of Canterbury}, Lab Instructor\\
& \small\textcolor{black!80}{ENCH281: Biology for Engineers}\\
Semester 2 2017 & \textbf{University of Canterbury}, Demonstrator (Teaching assistant)\\
& \small\textcolor{black!80}{BIOL213: Microbiology and Genetics}\\
Semester 1 2016 & \textbf{ETH}, Teaching assistant\\
& \small\textcolor{black!80}{551-0104-00L: Fundamentals of Biology II, Plant Physiology}\\
Semester 1 2014 & \textbf{Pontificia Universidad Católica de Chile}, Teaching assistant\\
& \small\textcolor{black!80}{BIO225C: Plant Physiology and Biochemistry}\\
& \small\textcolor{black!80}{BIO364C: Industrial Biotechnology}\\
Semester 2 2013 & \textbf{Pontificia Universidad Católica de Chile}, Teaching assistant\\
& \small\textcolor{black!80}{BIO266E: Laboratory of Biochemistry II: Molecular Genetics}\\
Semester 1 2013 & \textbf{Pontificia Universidad Católica de Chile}, Teaching assistant\\
& \small\textcolor{black!80}{BIO257C: Biochemistry}\\
& \small\textcolor{black!80}{BIO225C: Plant Physiology and Biochemistry}\\
Semester 1 2012 & \textbf{Pontificia Universidad Católica de Chile}, Teaching assistant\\
& \small\textcolor{black!80}{BIO257C: Biochemistry}\\
\end{tabular}

\subsection{Supervision}
~\begin{tabular}{p{3cm} p{14cm}}
2020--Present & Christian Stocks, M.Sc. student, Remus-Emsermann Lab, University of Canterbury (Co-supervision)\\
2020--2021 & Evan Kear, Undergraduate summer intern, Remus-Emsermann Lab, University of Canterbury\\
2019 & Christian Stocks, Undergraduate summer intern, Remus-Emsermann Lab, University of Canterbury\\
2016 & Michael Schl\"afli, Semester project student, Plant Cell Biology Group, ETH\\
2014 & Diego Bustos, Semester project student, Arce-Johnsoh Lab, Pontificia Universidad Católica de Chile
\end{tabular}

\subsection{Participation in Funded Projects}
~\begin{tabular}{p{3cm} p{14cm}}
2019--Present     &  \textbf{Bioprotection Core New Initiative Fund}, Associate Investigator\\
2017--Present     &  \textbf{Marsden Fast-Start Grant, Royal Society of New Zealand}, Ph.D. student\\
Apr 2014--Feb 2015     &  \textbf{Fundación para la Innovación Agraria}, Research Assistant\\
Jan 2013--May 2014     &  \textbf{Consorcio Tecnológico de la Fruta, ASOEX and Pontificia Universidad Católica de Chile}, Research student\\
\end{tabular}

\subsection{Reviewer Activity}
Phytobiomes, Basic and Applied Ecology, AMB Express

\section{Publications}

\printbibliography[category=Preprints, title = {Pre-prints}]
\printbibliography[category=Papers, title = {Peer-reviewed Papers}]
* Equal contribution
\printbibliography[category=Books, title = {Book Chapters}]

\section{Conference Participation}
~\begin{tabular}{p{3cm} p{14cm}}
2020 &  \textbf{Oral presentation}, New Zealand Microbiology Society (NZMS) Online Conference, New Zealand\\
     & \textbf{Oral presentation}, New Zealand Microbial Ecology Consortium Meeting (NZMEC) 6.0, Auckland, New Zealand\\
2019 & \textbf{Oral presentation}, NZMS Annual Conference, Palmerston North, New Zealand\\
     & \textbf{Oral presentation}, Canterbury Omics Symposium VIII, Christchurch, New Zealand\\
2018 & \textbf{Poster presentation}, NZMS-NZSBMB Joint Annual Conference, Dunedin, New Zealand\\
     & \textbf{Oral presentation}, Canterbury Omics Symposium VII, Christchurch, New Zealand\\
     & \textbf{Poster presentation}, NZMEC5.0, Auckland, New Zealand\\
Pre-2018 & \textbf{Poster presentation}, D-BIOL ETHZ Symposium IX, Davos, Switzerland \textbf{(2016)}\\
     & \textbf{Oral presentation}, Plant Biology Annual Conference IX, La Serena, Chile \textbf{(2014)}\\
     & \textbf{Poster presentation}, Panamerican Association for Biochemistry and Molecular Biology Congress XII, Puerto Varas, Chile \textbf{(2013)} \\
     & \textbf{Poster presentation}, International Symposium of Grapevine Physiology and Biotechnology IX, La Serena, Chile \textbf{(2013)}\\

\end{tabular}

\section{Technical Skills}
~\begin{tabular}{p{3cm} p{14cm}}
Programming     & R, Unix, \LaTeX\\
Imaging Processing & Photoshop, Illustrator, FIJI/ImageJ\\
Spatial statistics  & DAIME\\
Microscopy & Fluorescence, Confocal (Laser Scanning, Spinning Disk) Microscopy
\end{tabular}

\section{Languages}
~\begin{tabular}{p{3cm} p{14cm}}
Spanish     & Native proficiency\\
English     & Professional proficiency (C1 --- TOEFL score: 110, 2017)\\
German      & Basic proficiency (A2)
\end{tabular}

\section{References}

~\begin{tabular}{p{7.5cm} p{5cm} p{4cm}}
\multicolumn{3}{p{16.5cm}}{Prof. Mitja Remus-Emsermann, Ph.D.}\\
\small\textcolor{black!80}{\faIcon{university} Freie Universität Berlin, Berlin, Germany} & \small\textcolor{black!80}{\faIcon{envelope} m.remus-emsermann@fu-berlin.de} & \small\textcolor{black!80}{\faIcon{phone} +49 30 838 58031}\\ \\

\multicolumn{3}{p{16.5cm}}{Prof. Emerita Paula Jameson, ONZM, Ph.D.}\\
\small\textcolor{black!80}{\faIcon{university} University of Canterbury, Christchurch, New Zealand} & \small\textcolor{black!80}{\faIcon{envelope} paula.jameson@canterbury.ac.nz} & \small\textcolor{black!80}{\faIcon{phone} +64 33 69 5181}\\ \\

\multicolumn{3}{p{16.5cm}}{Assoc. Prof. Matthew Stott, Ph.D.}\\
\small\textcolor{black!80}{\faIcon{university} University of Canterbury, Christchurch, New Zealand} & \small\textcolor{black!80}{\faIcon{envelope} matthew.stott@canterbury.ac.nz} & \small\textcolor{black!80}{\faIcon{phone} +64 33 69 2511}\\
\end{tabular}

\end{document}